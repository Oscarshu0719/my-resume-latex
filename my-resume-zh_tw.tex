\documentclass{resume}

\usepackage[left=0.4 in,top=0.4in,right=0.4 in,bottom=0.4in]{geometry}
\usepackage{ctex}       %讓中英文字體分開設置
\usepackage{xcolor}
\usepackage{fontspec}
\defaultfontfeatures{Extension = .otf}
\usepackage{fontawesome}
\newcommand{\tab}[1]{\hspace{.2667\textwidth}\rlap{#1}} 
\newcommand{\itab}[1]{\hspace{0em}\rlap{#1}}

% \hypersetup{%
%   colorlinks=false,% hyperlinks will be black
%   linkbordercolor=red,% hyperlink borders will be red
%   pdfborderstyle={/S/U/W 1}% border style will be underline of width 1pt
% }

\name{許子駿}
\address{新北市, 臺灣} 
\address{
\small \href{tel:+886-987605719}{ \raisebox{-0.1\height}\faPhone\ \underline{+886-987605719} ~} \href{mailto:vm3y3rmp40719@gmail.com}{\raisebox{-0.2\height}\faEnvelope\  \underline{vm3y3rmp40719@gmail.com}} ~ 
  \href{https://www.linkedin.com/in/tzu-chun-hsu-ab4b3b188/}{\raisebox{-0.2\height}\faLinkedinSquare\ \underline{tzu-chun-hsu-ab4b3b188}}  ~
  \href{https://github.com/Oscarshu0719}{\raisebox{-0.2\height}\faGithub\ \underline{Oscarshu0719}}
}

\begin{document}

%----------------------------------------------------------------------------------------
% EDUCATION
%----------------------------------------------------------------------------------------
\vspace{-1em}
\begin{rSection}{學歷}

{\bf 浙江大學} \hfill {09 2016 -- 07 2020}\\
計算機科學與技術學士 \hfill \textit{杭州, 中國}\\
$\bullet$ 最後兩年 GPA: 3.30/4.00

{\bf 國立陽明交通大學} \hfill {09 2022 -- 08 2024 (預計)}\\
資訊科學與工程研究所甲組碩士 \hfill \textit{新竹, 臺灣}\\
$\bullet$ 第一年 GPA: 4.04/4.30

\end{rSection}

%----------------------------------------------------------------------------------------
% PROJECTS
%----------------------------------------------------------------------------------------
\vspace{-1em}
\begin{rSection}{專案}
\vspace{-1.25em}
    \item \textbf{\href{https://docs.google.com/presentation/d/1797idRRmgeD2JnUslGbflmGUO2X5lMY6/edit?usp=drive_link&ouid=101248488395326982475&rtpof=true&sd=true}{\textbf{\large{\underline{基於生成對抗網路的古箏自動編曲系統}}}}} \hfill {04 2018 -- 05 2019}\\
    \mbox{} \hfill \textit{杭州, 中國}
    \vspace{-0.5em}
    \begin{itemize}
        \item 基於變分自動編碼器 (VAE) 和生成對抗網路 (GAN),提出一個生成多音軌符號 (symbolic) 音樂的框架。
        \item 引入 WGAN-GP 解決模式崩壞 (mode collapse) 以及梯度消失 (vanishing gradient) 問題。
    \end{itemize}

    \item \textbf{\href{https://drive.google.com/file/d/1736XHtaeT58FL_gbOa9XjlyA6xAXDkg2/view?usp=drive_link}{\textbf{\large{\underline{基於生成對抗網路的語音轉換研究}}}}} \hfill {03 2020 -- 05 2020}\\
    \mbox{} \hfill \textit{杭州, 中國}
    \vspace{-0.5em}
    \begin{itemize}
        \item 在多語者 (multi-speaker) 非平行語料庫 (non-parallel corpus) 基礎上改進 StarGAN-VC2。
        \item 引入 WGAN-div, AdaIN 層和 WaveGlow 神經聲碼器 (neural vocoder),提升生成的語音品質。
    \end{itemize}
    
    \item \textbf{\href{https://drive.google.com/file/d/176xCWH6j0VAuoXLtB_HppxyZN-6ezq9_/view?usp=drive_link}{\textbf{\large{\underline{Synthetic Data Generation using Conditional Normalizing Flows}}}}} \hfill {03 2023 -- 05 2023}\\
    \mbox{} \hfill \textit{新竹, 臺灣}
    \vspace{-0.5em}
    \begin{itemize}
        \item 提出一個條件正規化流 (CNF) 模型,用於替換原始資料集,並生成新的資料集,同時能強調異常值 (outlier)。
        \item 建立一個瀏覽器-伺服器 (browser-server) 網路應用,用於視覺化結果。
    \end{itemize}

    \item \textbf{立體浮空影像品質補償演算法技術} \hfill {06 2023 -- 12 2023 (預計)}\\
    \mbox{} \hfill \textit{新竹, 臺灣}
    \vspace{-0.5em}
    \begin{itemize}
        \item 提出一個基於卷積神經網路 (CNN) 的浮空影像品質補償演算法,用於解決硬體缺陷問題,提升浮空影像的品質。
        \item 可以在不同的硬體上訓練和部署。
    \end{itemize}
\end{rSection} 

%----------------------------------------------------------------------------------------
% AWARDS	
%----------------------------------------------------------------------------------------
\vspace{-1em}
\begin{rSection}{獎項}
\vspace{-1.25em}
    \item \textbf{中國高校計算機大賽移動應用創新賽} \hfill {09 2018}\\
    {一等獎、最具創新獎} \hfill \textit{杭州, 中國}

    \item \textbf{第四屆中國” 互聯網 +” 大學生創新創業大賽} \hfill {09 2018}\\
    {金獎} \hfill \textit{杭州, 中國}

    \item \textbf{國家級大學生創新創業訓練計畫 (SRTP)} \hfill {04 2018 -- 05 2019}\\
    {優秀} \hfill \textit{杭州, 中國}
\end{rSection} 

%----------------------------------------------------------------------------------------
% EXPERIENCE	
%----------------------------------------------------------------------------------------

% \begin{rSection}{EXPERIENCE}
% \vspace{-1.25em}
% \item \textbf{The University of British Columbia Vancouver Summer Program} \hfill {07 2019 -- 08 2019}\\
% \mbox{} \hfill \textit{Vancouver, Canada}
%     \begin{itemize}
%         \item Included 2 courses, ”Algorithms and the World Wide Web” and ”Building Modern Web Applications”.
%         \item Collaborated with my teammates to complete our projects as team leader.
%     \end{itemize}
% \end{rSection} 

%----------------------------------------------------------------------------------------
% SKILLS	
%----------------------------------------------------------------------------------------
\vspace{-1em}
\begin{rSection}{技能}
\begin{tabular}{ @{} >{\bfseries}l @{\hspace{6ex}} l }
電腦語言 & Python, Java, C, JavaScript, SQL\\
技術/框架 & PyTorch, Flask, Spring Boot, Vue.js, Git, MySQL, MongoDB
\end{tabular}\\
\end{rSection}

\end{document}
