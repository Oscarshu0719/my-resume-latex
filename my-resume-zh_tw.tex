\documentclass[letterpaper,11pt,UTF8]{article}

\usepackage{ctex}
% \usepackage {xeCJK}
\usepackage{latexsym}
\usepackage[empty]{fullpage}
\usepackage{titlesec}
\usepackage{marvosym}
\usepackage[usenames,dvipsnames]{color}
\usepackage{verbatim}
\usepackage{enumitem}
\usepackage[hidelinks]{hyperref}
\usepackage[english]{babel}
\usepackage{tabularx}
\usepackage{fontawesome5}
\usepackage{multicol}
\usepackage{graphicx}
\setlength{\multicolsep}{-3.0pt}
\setlength{\columnsep}{-1pt}
% \input{glyphtounicode} % disable while using cTeX library

\RequirePackage{tikz}
\RequirePackage{xcolor}
\RequirePackage{fontawesome}
\usepackage{tikz}
\usetikzlibrary{svg.path}

% Disable bookmarks.
% \hypersetup{draft}

\definecolor{cvblue}{HTML}{0E5484}
\definecolor{black}{HTML}{130810}
\definecolor{darkcolor}{HTML}{0F4539}
\definecolor{cvgreen}{HTML}{3BD80D}
\definecolor{taggreen}{HTML}{00E278}
\definecolor{SlateGrey}{HTML}{2E2E2E}
\definecolor{LightGrey}{HTML}{666666}
\colorlet{name}{black}
\colorlet{tagline}{darkcolor}
\colorlet{heading}{darkcolor}
\colorlet{headingrule}{cvblue}
\colorlet{accent}{darkcolor}
\colorlet{emphasis}{SlateGrey}
\colorlet{body}{LightGrey}


%----------FONT OPTIONS----------
% sans-serif
% \usepackage[sfdefault]{FiraSans}
% \usepackage[sfdefault]{roboto}
% \usepackage[sfdefault]{noto-sans}
% \usepackage[default]{sourcesanspro}

% serif
% \usepackage{CormorantGaramond}
% \usepackage{charter}


% \pagestyle{fancy}
% \fancyhf{}  % clear all header and footer fields
% \fancyfoot{}
% \renewcommand{\headrulewidth}{0pt}
% \renewcommand{\footrulewidth}{0pt}

% Adjust margins
\addtolength{\oddsidemargin}{-0.6in}
\addtolength{\evensidemargin}{-0.5in}
\addtolength{\textwidth}{1.19in}
\addtolength{\topmargin}{-.7in}
\addtolength{\textheight}{1.4in}

\urlstyle{same}

\raggedbottom
\raggedright
\setlength{\tabcolsep}{0in}

% Sections formatting
\titleformat{\section}{
  \vspace{-4pt}\scshape\raggedright\large\bfseries
}{}{0em}{}[\color{black}\titlerule \vspace{-5pt}]

% Ensure that generate pdf is machine readable/ATS parsable (disable while using cTeX library)
% \pdfgentounicode=1


%-------------------------
% Custom commands
\newcommand{\resumeItem}[1]{
  \item\small{
    {#1 \vspace{-2pt}}
  }
}

\newcommand{\classesList}[4]{
    \item\small{
        {#1 #2 #3 #4 \vspace{-2pt}}
  }
}

\newcommand{\resumeSubheading}[4]{
  \vspace{-2pt}\item
    \begin{tabular*}{1.0\textwidth}[t]{l@{\extracolsep{\fill}}r}
      \textbf{\large#1} & \textbf{\small #2} \\
      \textit{\large#3} & \textit{\small #4} \\
      
    \end{tabular*}\vspace{-7pt}
}

\newcommand{\resumeSubSubheading}[2]{
    \item
    \begin{tabular*}{0.97\textwidth}{l@{\extracolsep{\fill}}r}
      \textit{\small#1} & \textit{\small #2} \\
    \end{tabular*}\vspace{-7pt}
}

\newcommand{\resumeProjectHeading}[2]{
    \item
    \begin{tabular*}{1.001\textwidth}{l@{\extracolsep{\fill}}r}
      \small#1 & \textbf{\small #2}\\
    \end{tabular*}\vspace{-7pt}
}

\newcommand{\resumeSubItem}[1]{\resumeItem{#1}\vspace{-4pt}}

\renewcommand\labelitemi{$\vcenter{\hbox{\tiny$\bullet$}}$}
\renewcommand\labelitemii{$\vcenter{\hbox{\tiny$\bullet$}}$}

\newcommand{\resumeSubHeadingListStart}{\begin{itemize}[leftmargin=0.0in, label={}]}
\newcommand{\resumeSubHeadingListEnd}{\end{itemize}}
\newcommand{\resumeItemListStart}{\begin{itemize}}
\newcommand{\resumeItemListEnd}{\end{itemize}\vspace{-5pt}}

\newcommand\sbullet[1][.5]{\mathbin{\vcenter{\hbox{\scalebox{#1}{$\bullet$}}}}}

%-------------------------------------------
%%%%%%  RESUME STARTS HERE  %%%%%%%%%%%%%%%%%%%%%%%%%%%%

\begin{document}

%----------HEADING----------
\begin{center}
  {\Huge \scshape \CJKfamily{zhkai} 許子駿} \\ \vspace{1pt}
  \heiti 新北市, 臺灣 \\ \vspace{8pt}
  \small \href{tel:+886-987605719}{ \raisebox{-0.1\height}\faPhone\ \underline{+886-987605719} ~} \href{mailto:vm3y3rmp40719@gmail.com}{\raisebox{-0.2\height}\faEnvelope\  \underline{vm3y3rmp40719@gmail.com}} ~ 
  \href{https://www.linkedin.com/in/tzu-chun-hsu-ab4b3b188/}{\raisebox{-0.2\height}\faLinkedinSquare\ \underline{tzu-chun-hsu-ab4b3b188}}  ~
  \href{https://github.com/Oscarshu0719}{\raisebox{-0.2\height}\faGithub\ \underline{Oscarshu0719}}
  % \vspace{-8pt}
\end{center}


%-----------EDUCATION-----------
\section{EDUCATION}
  \resumeSubHeadingListStart
    \resumeSubheading
      {浙江大學}{09 2016 -- 07 2020}
      {計算機科學與技術學士}{杭州, 中國}
      \resumeItemListStart
        \resumeItem{\normalsize{最後兩年GPA: 3.30/4.00}}
      \resumeItemListEnd
  \resumeSubHeadingListEnd
  

%-----------PROJECTS-----------
\section{PROJECTS}
  % \vspace{-5pt}
  \resumeSubHeadingListStart
    \resumeSubheading
      {\href{https://bit.ly/35GzBmc}{\textbf{\large{\underline{基於生成對抗網路的古箏自動編曲系統}}} \href{https://bit.ly/35GzBmc}{\raisebox{-0.1\height}\faExternalLink }}}{04 2018 -- 05 2019}
      {}{杭州, 中國}
      \resumeItemListStart
        \resumeItem{\normalsize{基於變分自動編碼器 (VAE) 和生成對抗網路 (GAN),提出一個生成多音軌 (multi-track) 符號音樂 (symbolic music) 的框架,並通過 \href{https://pytorch.org/}{PyTorch} 實現。}}
        \resumeItem{\normalsize{改進VAE,使得和弦更容易解析。}}
        \resumeItem{\normalsize{改進 \href{https://arxiv.org/abs/1704.00028}{WGAN-GP},解決模式崩壞 (mode collapse)、梯度消失 (vanishing gradient) 和部分音軌效果較差的問題。}}
        \resumeItem{\normalsize{提出三種損失函數 (loss function):\textit{element-wise}, \textit{discriminator dominant}, 和 \textit{hybrid}。}}
      \resumeItemListEnd 
      % \vspace{-13pt}

    \resumeSubheading
      {\href{https://bit.ly/3IzE3BU}{\textbf{\large{\underline{基於生成對抗網路的語音轉換研究}}} \href{https://bit.ly/3IzE3BU}{\raisebox{-0.1\height}\faExternalLink }}}{03 2020 -- 05 2020}
      {}{杭州, 中國}
      \resumeItemListStart
        \resumeItem{\normalsize{基於多語者 (multi-speaker) 非平行語料庫 (non-parallel corpus),改進 \href{https://arxiv.org/abs/1907.12279}{StarGAN-VC2}。}}
        \resumeItem{\normalsize{引入 \href{https://arxiv.org/abs/1712.01026}{WGAN-div} 解決 \href{https://arxiv.org/abs/1701.07875}{WGAN} 原有的k-Lipschitz限制。}}
        \resumeItem{\normalsize{通過將生成網路 (generator) 的 \href{https://arxiv.org/abs/1610.07629}{cIN} 層替換成 \href{https://arxiv.org/abs/1703.06868}{AdaIN} 層,提升生成的語音效果。}}
        \resumeItem{\normalsize{引入神經網路語音合成器 (neural vocoder) \href{https://arxiv.org/abs/1811.00002}{WaveGlow},通過梅爾時頻譜 (mel-spectrograms) 生成較高音質的語音。}}
      \resumeItemListEnd 
      % \vspace{-13pt}
  \resumeSubHeadingListEnd
\vspace{-12pt}


%-----------AWARDS-----------
\section{AWARDS}
  \resumeSubHeadingListStart
    \resumeSubheading
      {中國高校計算機大賽移動應用創新賽}{09 2018}
      {一等獎}{杭州, 中國}

    \resumeSubheading
      {第四屆中國"互聯網+"大學生創新創業大賽}{09 2018}
      {金獎}{杭州, 中國}

    \resumeSubheading
      {港澳台學生獎學金}{09 2018}
      {三等獎}{杭州, 中國}

    \resumeSubheading
      {國家級大學生創新創業訓練計畫 (SRTP)}{04 2018 -- 05 2019}
      {優秀}{杭州, 中國}

    \resumeSubheading
      {港澳台學生獎學金}{12 2019}
      {二等獎}{杭州, 中國}
  \resumeSubHeadingListEnd
\vspace{-12pt}


% {Company Name \href{certificate Link}{\raisebox{-0.1\height}\faExternalLink }}{MM YYYY -- MM YYYY} 
% {\underline{Role Name}}{city, country}
%-----------EXPERIENCE-----------

% Omit this part to make resume in 1 paper.

% \section{EXPERIENCE}
%   \resumeSubHeadingListStart
%     \resumeSubheading
%       {英屬哥倫比亞大學 (UBC) 溫哥華暑期計畫 (VSP)}{07 2019 -- 08 2019}
%       {}{溫哥華, 加拿大}
%       \resumeItemListStart
%         \resumeItem{\normalsize{高強度的四周課程,包含"Algorithms and the World Wide Web"和"Building Modern Web Applications"。}}
%         \resumeItem{\normalsize{在團隊中擔任組長,並和組員共同完成專案。}}
%         \resumeItem{\normalsize{在這四周感受UBC不同的學習氛圍,拓展視野。}}
%       \resumeItemListEnd
%   \resumeSubHeadingListEnd
% \vspace{-12pt}


%-----------PROGRAMMING SKILLS-----------
\section{TECHNICAL SKILLS}
  \resumeItemListStart
    \resumeItem{\textbf{\normalsize{語言:}}{ \normalsize{\href{https://www.python.org/}{Python}, \href{https://www.java.com/}{Java}, C, JavaScript}}}
    \resumeItem{\textbf{\normalsize{技術/框架:}}{ \normalsize{\href{https://pytorch.org/}{PyTorch}, \href{https://keras.io/}{Keras}, \href{https://opencv.org/}{OpenCV}, \href{https://spring.io/projects/spring-boot}{Spring Boot}, \href{https://vuejs.org/}{Vue.js}, \href{https://www.mysql.com/}{MySQL}, \href{https://flask.palletsprojects.com/}{Flask}, \href{https://nodejs.org/}{Node.js}, \href{https://www.mathworks.com/products/matlab.html}{MATLAB}, \href{https://git-scm.com/}{Git}, \href{https://www.mongodb.com/}{MongoDB}, \href{https://www.latex-project.org/}{\LaTeX}}}}
  \resumeItemListEnd 


%-----------INVOLVEMENT---------------
% \section{EXTRACURRICULAR}
%     \resumeSubHeadingListStart
%         \resumeSubheading{Organization Name \href{Certificate Proof link}{\raisebox{-0.1\height}\faExternalLink } }{MM YYYY -- MM YYYY}{\underline{Role}}{Location}
%             \resumeItemListStart
%                 \resumeItem{\normalsize{About the role \textbf{and responsibilities carried out.}}}
%                 \resumeItem{\normalsize{Participation Certificate. \href{ParticipationCertificateLink.com}{\raisebox{-0.1\height}\faExternalLink }}}
%             \resumeItemListEnd
%     \resumeSubHeadingListEnd
%  \vspace{-11pt}
 
%-----------CERTIFICATIONS---------------
% \section{CERTIFICATIONS}

% $\sbullet[.75] \hspace{0.1cm}$ {\href{certificateLink.com}{ReactJS \& Redux - Udemy}} \hspace{1.6cm}
% $\sbullet[.75] \hspace{0.1cm}$ {\href{certificateLink.com}{Java}} \hspace{2.59cm}
% $\sbullet[.75] \hspace{0.2cm}${\href{certificateLink.com} {Command Line in Linux - Coursera}}\\

% $\sbullet[.75] \hspace{0.2cm}${\href{certificateLink.com}{Python for Data Science - XIE}} \hspace{1cm}
% $\sbullet[.75] \hspace{0.1cm}$ {\href{certificateLink.com}{SQL}} \hspace{2.6cm}
% $\sbullet[.75] \hspace{0.2cm}${\href{certificateLink.com}{Microsoft AI Classroom - Microsoft}} \\

% $\sbullet[.75] \hspace{0.2cm}${\href{certificateLink.com}{\textbf{5 Stars} in \textbf{C++} \& \textbf{SQL} \href{certificateLink.com}{\raisebox{-0.1\height}\faExternalLink }}}\hspace{1.45cm}
% $\sbullet[.75] \hspace{0.2cm}${\href{certificateLink.com}{MongoDB Basics}} \hspace{0.5cm}
% $\sbullet[.75] \hspace{0.2cm}${\href{certificateLink.com}{NodeJS with Express \& MongoDB - Udemy}} \\


\end{document}