\documentclass{resume}

\usepackage[left=0.4 in,top=0.4in,right=0.4 in,bottom=0.4in]{geometry}
\usepackage{xcolor}
\usepackage{fontspec}
\defaultfontfeatures{Extension = .otf}
\usepackage{fontawesome}
\newcommand{\tab}[1]{\hspace{.2667\textwidth}\rlap{#1}} 
\newcommand{\itab}[1]{\hspace{0em}\rlap{#1}}

% \hypersetup{%
%   colorlinks=false,% hyperlinks will be black
%   linkbordercolor=red,% hyperlink borders will be red
%   pdfborderstyle={/S/U/W 1}% border style will be underline of width 1pt
% }

\name{Tzu Chun, Hsu}
\address{New Taipei City, Taiwan} 
\address{
\small \href{tel:+886-987605719}{ \raisebox{-0.1\height}\faPhone\ \underline{+886-987605719} ~} \href{mailto:vm3y3rmp40719@gmail.com}{\raisebox{-0.2\height}\faEnvelope\  \underline{vm3y3rmp40719@gmail.com}} ~ 
  \href{https://www.linkedin.com/in/tzu-chun-hsu-ab4b3b188/}{\raisebox{-0.2\height}\faLinkedinSquare\ \underline{tzu-chun-hsu-ab4b3b188}}  ~
  \href{https://github.com/Oscarshu0719}{\raisebox{-0.2\height}\faGithub\ \underline{Oscarshu0719}}
}

\begin{document}

%----------------------------------------------------------------------------------------
% WORK EXPERIENCE
%----------------------------------------------------------------------------------------
\vspace{-1em}
\begin{rSection}{Work Experience}
    {\bf Camera ISP Engineer} \hfill {02 2025 -- present}\\
    {AI camera algorithm development @ Realtek Semiconductor Corp.} \hfill \textit{Hsinchu, Taiwan}\\
    \begin{itemize}
        \item ISP Auto Verification, CModel, and SDK development.
    \end{itemize}
\end{rSection}

%----------------------------------------------------------------------------------------
% EDUCATION
%----------------------------------------------------------------------------------------
\vspace{-1em}
\begin{rSection}{Education}

{\bf Zhejiang University} \hfill {09 2016 -- 07 2020}\\
{Bachelor of Engineering in Computer Science and Technology} \hfill \textit{Hangzhou, China}\\
$\bullet$ Last two years GPA: 3.30/4.00

{\bf National Yang Ming Chiao Tung University} \hfill {09 2022 -- 02 2025}\\
{Master of Science in Computer Science and Engineering} \hfill \textit{Hsinchu, Taiwan}\\
$\bullet$ Overall GPA: 4.19/4.30

\end{rSection}

%----------------------------------------------------------------------------------------
% PROJECTS
%----------------------------------------------------------------------------------------
\vspace{-1em}
\begin{rSection}{PROJECTS}
\vspace{-1.25em}
    \item \textbf{\href{https://docs.google.com/presentation/d/1797idRRmgeD2JnUslGbflmGUO2X5lMY6/edit?usp=drive_link&ouid=101248488395326982475&rtpof=true&sd=true}{\textbf{\large{\underline{Chord learning and adversarial framework for symbolic music generation}}}}} \hfill {04 2018 -- 05 2019}\\
    \mbox{} \hfill \textit{Hangzhou, China}
    \begin{itemize}
        \item Proposed a chord learning framework for multi-track symbolic music generation\\based on VAE and GAN.
        % \item Improved VAE to extract chords more easily.
        \item Introduced WGAN-GP to solve mode collapse and vanishing gradient problem.
        % \item Proposed three distinct loss functions: \textit{element-wise}, \textit{discriminator dominant}, and \textit{hybrid}.
    \end{itemize}

    \item \textbf{\href{https://drive.google.com/file/d/1736XHtaeT58FL_gbOa9XjlyA6xAXDkg2/view?usp=drive_link}{\textbf{\large{\underline{Voice Conversion Based on Generative Adversarial Networks}}}}} \hfill {03 2020 -- 05 2020}\\
    \mbox{} \hfill \textit{Hangzhou, China}
    \begin{itemize}
        \item Improved StarGAN-VC2 based on multi-speaker non-parallel corpus.
        \item Introduced WGAN-div, AdaIN layer, and neural vocoder, WaveGlow,\\for better generated audio quality.
        % \item Introduced WGAN-div to resolve k-Lipschitz constraint of WGAN.
        % \item Replaced cIN layer with trainable AdaIN layer for better performance.
        % \item Introduced neural vocoder, WaveGlow, to generate high quality speech from mel-spectrograms.
    \end{itemize}
    
    % \item \textbf{\href{https://drive.google.com/file/d/176xCWH6j0VAuoXLtB_HppxyZN-6ezq9_/view?usp=drive_link}{\textbf{\large{\underline{Synthetic Data Generation using Conditional Normalizing Flows}}}}} \hfill {03 2023 -- 05 2023}\\
    % \mbox{} \hfill \textit{Hsinchu, Taiwan}
    % \begin{itemize}
    %     \item Ability to replace original datasets and maintain same data distribution, and also\\
    %     highlight outliers for analysis.
    %     \item Built a browser-server web application to visualize results.
    % \end{itemize}

    \item \textbf{Floating image quality compensation algorithm technology} \hfill {08 2023 -- 02 2025}\\
    \mbox{} \hfill \textit{Hsinchu, Taiwan}
    \begin{itemize}
        \item Enhanced image quality of floating images degraded by hardware imperfections.
        \item Proposed a two-stage method: (1) simulating the imaging process by solving a QP problem derived from optical principles, and (2) refining light field images through iterative optimization based on the simulation.
    \end{itemize}
\end{rSection} 

%----------------------------------------------------------------------------------------
% AWARDS	
%----------------------------------------------------------------------------------------
\vspace{-1em}
\begin{rSection}{AWARDS}
\vspace{-1.25em}
    \item \textbf{CCCC-Mobile Application Innovation Contest} \hfill {09 2018}\\
    {First Prize and Most Innovative Award} \hfill \textit{Hangzhou, China}

    \item \textbf{The 4th ”Internet+” Innovation and Entrepreneurship Competition} \hfill {09 2018}\\
    {Gold Award} \hfill \textit{Hangzhou, China}

    % \item \textbf{College Students’ Innovative Entrepreneurial Training Plan Program (SRTP)} \hfill {04 2018 -- 05 2019}\\
    % {Excellent} \hfill \textit{Hangzhou, China}
\end{rSection} 

%----------------------------------------------------------------------------------------
% EXPERIENCE	
%----------------------------------------------------------------------------------------
% \vspace{-1em}
% \begin{rSection}{EXPERIENCE}
% \vspace{-1.25em}
% \item \textbf{The University of British Columbia Vancouver Summer Program} \hfill {07 2019 -- 08 2019}\\
% \mbox{} \hfill \textit{Vancouver, Canada}
%     \begin{itemize}
%         \item Included 2 courses, ”Algorithms and the World Wide Web” and ”Building Modern Web Applications”.
%         \item Collaborated with my teammates to complete our projects as team leader.
%     \end{itemize}
% \end{rSection} 

%----------------------------------------------------------------------------------------
% SKILLS	
%----------------------------------------------------------------------------------------
\vspace{-1em}
\begin{rSection}{SKILLS}
\begin{tabular}{ @{} >{\bfseries}l @{\hspace{6ex}} l }
Languages & Python, Java, C, JavaScript, SQL\\
Technologies/Frameworks & PyTorch, Flask, Spring Boot, Vue.js, Git, MySQL, MongoDB
\end{tabular}\\
\end{rSection}

\end{document}
