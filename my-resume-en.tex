\documentclass{resume} % Use the custom resume.cls style

\usepackage[left=0.4 in,top=0.4in,right=0.4 in,bottom=0.4in]{geometry} % Document margins
\usepackage{xcolor}
\newcommand{\tab}[1]{\hspace{.2667\textwidth}\rlap{#1}} 
\newcommand{\itab}[1]{\hspace{0em}\rlap{#1}}

% \hypersetup{%
%   colorlinks=false,% hyperlinks will be black
%   linkbordercolor=red,% hyperlink borders will be red
%   pdfborderstyle={/S/U/W 1}% border style will be underline of width 1pt
% }

\name{Tzu Chun, Hsu} % Your name
\address{New Taipei City, Taiwan} 
\address{\href{tel:+886-987605719}{+886-987605719} \\ \href{mailto:vm3y3rmp40719@gmail.com}{vm3y3rmp40719@gmail.com} \\ \href{https://www.linkedin.com/in/tzu-chun-hsu-ab4b3b188/}{LinkedIn} \\ \href{https://github.com/Oscarshu0719}{GitHub}}  %

\begin{document}

%----------------------------------------------------------------------------------------
%	EDUCATION SECTION
%----------------------------------------------------------------------------------------

\begin{rSection}{Education}

{\bf Zhejiang University} \hfill {09 2016 – 07 2020}\\
Bachelor of Engineering in Computer Science and Technology \hfill \textit{Hangzhou, China}
\begin{itemize}
    \item Last two years cumulative GPA: 3.30/4.00
\end{itemize}

{\bf National Yang Ming Chiao Tung University} \hfill {09 2022 – 06 2024 (Expected)}\\
Master of Science in Computer Science and Engineering \hfill \textit{Hsinchu, Taiwan}
\begin{itemize}
    \item First semester GPA: 4.15/4.30
\end{itemize}

\end{rSection}

%----------------------------------------------------------------------------------------
%	PROJECTS SECTION
%----------------------------------------------------------------------------------------

\begin{rSection}{PROJECTS}
\vspace{-1.25em}
\item \textbf{\href{https://docs.google.com/presentation/d/1ge2It3UsAvTwpAk-9LUnmiz7dRqJnpFX/edit?usp=sharing&ouid=101248488395326982475&rtpof=true&sd=true}{\textbf{\large{\underline{Chord learning and adversarial framework for symbolic music generation}}}}} \hfill {04 2018 – 05 2019}\\\mbox{} \hfill \textit{Hangzhou, China}
\begin{itemize}
    \item Proposed a chord learning framework for multi-track symbolic music generation based on VAE and GAN.
    % \item Improved VAE to extract chords more easily.
    \item Introduced WGAN-GP to solve mode collapse and vanishing gradient problem.
    % \item Proposed three distinct loss functions: \textit{element-wise}, \textit{discriminator dominant}, and \textit{hybrid}.
\end{itemize}
\item \textbf{\href{https://drive.google.com/file/d/1QqgPoRGhaSeS0QWrjnR8mxhvV_onwZ5y/view?usp=drive_link}{\textbf{\large{\underline{Voice Conversion Based on Generative Adversarial Networks}}}}} \hfill {03 2020 -- 05 2020}\\\mbox{} \hfill \textit{Hangzhou, China}
\begin{itemize}
    \item Improved StarGAN-VC2 based on multi-speaker non-parallel corpus.
    \item Introduced WGAN-div, AdaIN layer, and neural vocoder, WaveGlow, for better results.
    % \item Introduced WGAN-div to resolve k-Lipschitz constraint of WGAN.
    % \item Replaced cIN layer with trainable AdaIN layer for better performance.
    % \item Introduced neural vocoder, WaveGlow, to generate high quality speech from mel-spectrograms.
\end{itemize}
\item \textbf{Synthetic Data Generation using Conditional Normalizing Flows} \hfill {03 2023 -- 05 2023}\\\mbox{} \hfill \textit{Hsinchu, Taiwan}
\begin{itemize}
    \item Ability to replace original datasets and maintain same data distribution, and also highlight outliers to analyze.
    \item Built an B/S application to visualize datasets.
\end{itemize}
\end{rSection} 

%----------------------------------------------------------------------------------------
% AWARDS STRENGTHS	
%----------------------------------------------------------------------------------------

\begin{rSection}{AWARDS}
\vspace{-1.25em}
\item \textbf{CCCC-Mobile Application Innovation Contest} \hfill {09 2018}\\{First Prize} \hfill \textit{Hangzhou, China}
\item \textbf{The 4th ”Internet+” Innovation and Entrepreneurship Competition} \hfill {09 2018}\\{Gold Award} \hfill \textit{Hangzhou, China}
\item \textbf{College Students’ Innovative Entrepreneurial Training Plan Program (SRTP)} \hfill {04 2018 – 05 2019}\\{Excellent} \hfill \textit{Hangzhou, China}
\end{rSection} 

%----------------------------------------------------------------------------------------
% EXPERIENCE STRENGTHS	
%----------------------------------------------------------------------------------------

\begin{rSection}{EXPERIENCE}
\vspace{-1.25em}
\item \textbf{The University of British Columbia Vancouver Summer Program} \hfill {07 2019 – 08 2019}\\\mbox{} \hfill \textit{Vancouver, Canada}
\begin{itemize}
    \item Included 2 courses, ”Algorithms and the World Wide Web” and ”Building Modern Web Applications”.
    \item Collaborated with my teammates to complete our projects as team leader.
\end{itemize}
\end{rSection} 

\begin{rSection}{SKILLS}

\begin{tabular}{ @{} >{\bfseries}l @{\hspace{6ex}} l }
Languages & Python, Java, C, JavaScript\\
Technologies/Frameworks & PyTorch, Flask, Spring Boot, Vue.js, Git, SQL, MongoDB
\end{tabular}\\
\end{rSection}

\end{document}
